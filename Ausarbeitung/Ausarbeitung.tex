
\documentclass[course=erap]{aspdoc}
\graphicspath{ {./images/} }
\usepackage{algorithm}
\usepackage{algpseudocode}
\usepackage{amsmath,amsthm}
\usepackage{amssymb}
\usepackage[section]{placeins}
\usepackage[font=small,labelfont=bf]{caption}
\usepackage{subfig}
\newcommand{\uproman}[1]{\uppercase\expandafter{\romannumeral#1}}
\newcommand{\lowroman}[1]{\romannumeral#1\relax}
\newcommand{\theGroup}{142}
\newcommand{\theNumber}{A213} 
\author{Anton Baumann \and Felix Brandis \and Michal Cizevskij}
\date{Sommersemester 2020}  

% Diese Zeile bitte -nicht- aendern.
\title{Gruppe \theGroup{} -- Abgabe zu Aufgabe \theNumber}



\begin{document}
\maketitle

\section{Einleitung}
Im Rahmen des Praktikums Rechnerarchitektur hat sich unsere Gruppe mit dem Thema der Moore-Kurve auseinandergesetzt. Die Moore-Kurve ist eine von dem Mathematiker Eliakim Hastings Moore in 1900 entwickelte raumfüllende Kurve. Sie basiert auf der Hilbert-Kurve und findet oft Anwendung in Bildverarbeitung und Datendarstellung. Unsere Aufgabe war es, einen Algorithmus in Assembler zu entwickeln, der die Koordinaten einer Moore-Kurve beliebigen Grades berechnet. Hierbei sollten wir auch beantworten, ob es möglich ist, einen beliebigen Punkt der Kurve in konstanter Zeit unabhängig vom Grad zu berechnen.

\begin{figure}[h]
\centering
    \includegraphics[width=8cm, height=5cm]{Moore_and_Hilbert}
    \caption{Die drei ersten Iterationen der Hilbert- und Moore-Kurve \cite{pic:hilbert}, \cite{pic:Moore}}
    \label{fig:mooreCurve}
\end{figure}

\subsection{Charakterisierung der Moore-Kurve und weiteren raumfüllenden Kurven}
Raumfüllende Kurven sind Linien, die jeden Punkt des Intervalls $[0,1]^d$ durchlaufen. Diese können im zweidimensionalen Raum als surjektive Abbildungen dargestellt werden, welche vom Einheitsintervall $[0,1]$ in das Einheitsquadrat $[0,1]^2$ abbilden. 
\newline
Eine wichtige Unterkategorie der raumfüllenden Kurven ist FASS („space-filling, self-avoiding, simple and self-similar“). Solche Kurven  erfüllen viele oft in der Praxis geforderte Bedingungen: Zu einem ist jede FASS-Kurve selbst ausweichend. Das bedeutet, dass jeder Punkt exakt einmal durchlaufen werden muss. Außerdem behalten die FASS-Kurven die Lokalität der Punkte bei. Sind Punkte räumlich nah im eindimensionalen Einheitsintervall, so gilt dies ebenfalls für ihr Bild in $[0,1]^d$ (siehe Abb. \ref{fig:locality}). 

\begin{figure}[h]
\centering
      \includegraphics[width=6cm, height=4cm]{Locality}
    \caption{Lokalität einer raumfüllenden Kurve \cite{pic:locality}}
    \label{fig:locality}
\end{figure}

Eine sehr bekannte Kurve, die die FASS-Eigenschaften erfüllt, ist die Hilbert-Kurve. Diese wurde von dem gleichnamigen Mathematiker David Hilbert in 1891 entwickelt und findet aufgrund ihres einfachen Aufbaus und ihrer Stetigkeit besonders oft in der Praxis Verwendung.
\newline
Die Moore-Kurve ist eine Variation der Hilbert-Kurve, welche sich aus vier rotierten Hilbert-Kurven geringeren Grades zusammensetzt. 
Der wesentliche Unterschied zwischen den beiden Kurven ist, dass die Moore-Kurve im Gegensatz zur Hilbert-Kurve eine geschlossene Form hat, Anfangs- und Endpunkt also benachbart sind.
\\
In raumbezogenen Anwendungen kommt es oft vor, dass raumfüllende Kurven verwendet werden, um effizient geographische Daten, die in zwei- oder dreidimensionaler Form vorhanden sind, auf eindimensionale Indizes zu abzubilden. Hierfür wird das Gebiet in Zellen aufgeteilt, welche zu einer kompakten Darstellung der bestimmten Region dienen. Dabei ist es erwünscht, dass geographische Nähe auch in der Abbildung auf den eindimensionalen Raum erhalten bleibt. Benachbarte Bilddaten liegen somit nebeneinander im Speicher und können daher effizienter geladen werden.

\begin{figure}[h]
    \centering
    \begin{minipage}{0.45\textwidth}
        \centering
        \includegraphics[width=0.8\textwidth]{Map}\\
        \caption{Einteilung von New York in Zellen \cite{Kreiss2016}}
        \label{fig:Gmaps}
    \end{minipage}\hfill
    \begin{minipage}{0.45\textwidth}
        \centering
        \includegraphics[width=0.9\textwidth]{Earth} \\
        \caption{Hilbert-Kurve projiziert auf die Erde \cite{Kreiss2016}}
        \label{fig:hilbertCurveOnEarth}
    \end{minipage}
\end{figure}

Da eine FASS-Kurve genau diese Anforderungen erfüllt, eignet sie sich besonders gut für solch eine Verwendung. Ein konkretes Beispiel der Anwendung wäre die \href{https://code.google.com/archive/p/s2-geometry-library/}{S2-geometry-library von Google}. 

\subsection{Benutzerdokumentation}
Der Quellcode unseres Programms enthält ein Makefile. Somit kann auf einem Computer, der \textit{AVX2} unterstützt der Code durch einen Aufruf von \texttt{make} kompiliert werden. Auf älteren Maschinen kann mit \texttt{make disable\_avx} gebuildet werden.
\newline
Um eine Moore-Kurve \textit{6. Grades} mithilfe der Assembler-Implementierung zu berechnen und unter \textit{moore.svg} abzuspeichern kann man nun die Binärdatei mit \texttt{./moore -{}-implementation asm -{}-degree 6 -{}-path moore.svg} ausführen. 
Um die verschiedenen Implementierungen zu benchmarken kann das Programm wie folgt im Benchmark-Modus aufgerufen werden: \texttt{./moore -{}-benchmark -{}-degree 6 -{}-repetitions 1000 -{}-write\_results}. Da hier die \textit{write results} Flag gesetzt ist, wird für jede getestete Implementierung das Ergebnis der letzten Iteration als \textit{.svg} Datei gespeichert. 

\section{Lösungsansatz}

\subsection{Die Moore-Kurve}

Wie im letzten Abschnitt erwähnt, kann die Moore-Kurve des Grads $n$ aus vier Kopien der Hilbert-Kurve des Grads $n-1$ erzeugt werden (für $n > 1$), siehe dazu auch Abbildung \ref{HilbertZuMoore}.
Das Problem kann also heruntergebrochen werden auf die Erzeugung der Hilbert-Kurve und dem anschließendenden Kopieren, Verschieben und Rotieren der Punkte. 
\\
Es gibt rekursive Algorithmen für die Erzeugung der Hilbert-Kurve, diese klammern wir im Folgenden jedoch aus. (Siehe Aufgabenstellung)
\FloatBarrier
\subsection{Erzeugung der Hilbert-Kurve: iterativer Punkt-für-Punkt Ansatz}
Im Folgenden soll ein Algorithmus dargestellt werden, der für gegebenen Index $i$ und Grad $n$ die Koordinaten des korrespondierenden Punktes an der Stelle $i$ der Moore-Kurve berechnet.
Geht man davon aus, schon einen solchen funktionierenden Algorithmus für die Hilbert-Kurve zu haben, so muss nur der Quadrant des Punktes auf der Moore-Kurve bestimmt werden, dann der entsprechende Punkt der Hilbert-Kurve errechnet und je nach Quadrant verschoben, gespiegelt oder rotiert werden. Dies wird in Abbildung \ref{HilbertZuMoore} grafisch veranschaulicht und im Algorithmus \ref{moore:iterative} (alle Algorithmen im Anhang) skizziert.
\begin{figure}[h]
\centering
\def\svgwidth{240pt}
\input{images/HilbertZuMooreVektorG.pdf_tex}
\caption{Transformation von Hilbert-Kurve des Grads n-1 zu Moore-Kurve des Grads n}
\label{HilbertZuMoore}
\end{figure}
Damit Algorithmus \ref{moore:iterative}, der einen erzeugten Hilbert-Punkt von Grad $n-1$ nur noch verschiebt, funktioniert, fehlt noch eine Methode, die für gegebenen Index $i$ und Grad $n$ die Koordinaten des entsprechenden Punktes auf der Hilbert-Kurve zurückgibt. \\
Abbildung \ref{fig:hilbertBuildup} zeigt, wie das funktioniert: Für jeden Punkt beginnt man mit den Koordinaten $(0, 0)$. Wird der Index in Binärdarstellung umgewandelt, so können mit den least significant Bits beginnend immer zwei Stellen angesehen werden und dann nach dem in der Graphik dargestellten Entscheidungsmuster die Koordinaten durch Spiegelung und Verschiebung manipuliert werden. Durch iterative Ausführung und mit jeder Iteration einer Vergrößerung des Offsets können so die Koordinaten jedes Punktes bestimmt werden. \cite{Chwedzuk2016} Algorithmus \ref{hilbert:iterative} zeigt eine mögliche Formalisierung.

\begin{figure}[h]
    \centering
    \subfloat{{\includegraphics[width=1.0\textwidth]{HilbertToHilbert} }}%
    \caption{Bottom-up Berechnung der Koordinaten eines Punkts der Hilbert-Kurve, Beispiel für Punkt mit Index 54 und Hilbert-Kurve des Grads 3 (Adaptiert von \cite{Chwedzuk2016})}%
    \label{fig:hilbertBuildup}%
\end{figure}






\FloatBarrier
\subsection{Erzeugung der Hilbert-Kurve: Dynamischer Ansatz}
Der oben dargestellte Algorithmus hat den Vorteil, für einen gegebenen Index und Grad die Koordinaten des korrespondierenden Punktes ausrechnen zu können, ohne dafür die gesamte Kurve erzeugen zu müssen.
Soll allerdings die gesamte Kurve erstellt werden, also x- und y-Werte für alle Punkte errechnet werden, so ist der iterative Aufruf des oben beschriebenen Algorithmus für jeden Index sehr ineffizient.\\
Ebenso wie die Moore-Kurve des Grads $n$ aus vier Kopien der Hilbert-Kurve des Grads $n-1$ aufgebaut werden kann, kann auch die Hilbert-Kurve des Grads $n$ aus vier Kopien der Hilbert-Kurve des Grads $n-1$ erzeugt werden (Abbildung \ref{HilbertZuHilbert}).
\begin{figure}[h]
\centering
\def\svgwidth{240pt}
\input{images/HilbertZuHilbertVektorG.pdf_tex}
\caption{Transformation von Hilbert-Kurve des Grads $n-1$ zu Hilbert-Kurve des Grads $n$}
\label{HilbertZuHilbert}
\end{figure}
Man kann also bei der Erzeugung der Hilbert-Kurve ähnlich vorgehen wie beim Algorithmus \ref{moore:iterative}, der aus der Hilbert-Kurve des Grads $n-1$ die Moore-Kurve des Grads $n$ erzeugt: Mit gegebenen Punkten der Hilbert-Kurve des Grads $n-1$ lassen sich alle Punkte der Hilbert-Kurve des Grads $n$ erzeugen. 
Die Spiegelungen, Rotationen und Translationen für die jeweiligen Quadranten sind natürlich anders als in Algorithmus \ref{moore:iterative}. \\
Dieser Ansatz kann zwar keinen einzelnen Punkt an einem gegebenen Index liefern ohne die gesamte Kurve zu berechnen,  die Berechnung aller Punkte der Kurve ist dafür aber bei weitem effizienter als der iterative Aufruf des Punkt-zu-Punkt Ansatzes, da Ergebnisse schon durchgeführter Berechnungen wiederverwertet werden. Algorithmus \ref{moore:batch} zeigt, wie zunächst die Hilbert-Kurve bis Grad $n-1$ aufgebaut und daraus dann die Moore-Kurve des Grads $n$ erzeugt wird. Für die Nachvollziehbarkeit sind Abbildung \ref{HilbertZuMoore} und \ref{HilbertZuHilbert} essentiell.


\FloatBarrier
\subsection{Theoretische Laufzeitanalyse und Vergleich}
Man kann sehen, dass der Aufwand für die Berechnung eines einzelnen Punktes beim Punkt-für-Punkt Algorithmus (Algorithmus \ref{hilbert:iterative}) vom Grad der Kurve abhängt: In der Unterfunktion $berechneHilbertKoordinate(index, degree)$  wird eine Schleife $log_2(2^{degree})-1 = degree -1$ mal durchlaufen. Da der Aufwand der Operationen innerhalb der Schleife konstant ist, steigt der Rechenaufwand pro Punkt in Algorithmus \ref{hilbert:iterative} also linear mit dem Grad der Kurve.
\newline
Für Algorithmus \ref{moore:batch} hingegen lässt sich zeigen, dass der durchschnittliche Aufwand für die Berechnung eines Punktes konstant ist, unabhängig vom Grad der Kurve. \\
Algorithmus \ref{moore:batch} besteht aus dem einmaligen Aufruf der Hilbert-Unterfunktion und dann dem Verschieben der erhaltenen Punkte zur Erzeugung der Moore-Kurve. \\
In der Hilbert-Unterfunktion gibt es zwei ineinander verschachtelte Schleifen: eine, die die Indexvariable $j$ von $2$ bis inklusiv  $degree -1$ durchläuft (Denn die Hilbert-Unterfunktion wird ja mit $degree-1$ aufgerufen) und die innere Schleife, bei der die Punkte des letzten Grades kopiert und verschoben werden. \\ 
Diese innere Schleife wird $quarter = 4^{j-1}$ mal durchlaufen, da immer alle Punkte der Kurve des Grads $j-1$ transformiert werden müssen, und sie hat konstanten Aufwand $c$. In ihr wird ein Punkt aus dem Speicher  geladen, die Koordinaten werden manipuliert und an 4 Stellen wieder in den Speicher geschrieben. (Die Annahme der Konstanz ist hier natürlich theoretisch: sie berücksichtigt keine Zeitunterschiede durch Caching oder RAM-Limits, sondern geht von einem idealisierten Maschinenmodell ohne die Probleme des Von-Neumann-Bottlenecks aus)\\
Nach Durchlaufen der Hilbert-Unterfunktion werden in der Moore-Funktion dann noch einmal  $4^{degree-1}$ Punkte kopiert und an die richtige Stelle geschoben. \\ Der Aufwand für die Berechnung aller Punkte der Moore-Kurve des Grads $n$ $Op(n)$ kann also folgendermaßen rekursiv definiert werden:
\begin{align*}
\uproman{1}: Op(n) &= Op(n-1) + 4^{n-1}\cdot c \\
\uproman{2}: Op(1) &= c
\end{align*}
Diese rekursive Gleichung kann nun aufgelöst werden: 
\begin{align*}
  Op(n) &= Op(n-1) + 4^{n-1}\cdot c\\
    &= Op(n-2) + 4^{n-2}\cdot c + 4^{n-1}\cdot c \\
    &= c\cdot(4^{0} + ... + 4^{n-2} + 4^{n-1}) \\
    &= c\cdot\sum_{i=0}^{n-1} 4^{i}\\
    &= c\cdot\frac{1}{3}(4^n-1)\\
\end{align*}
Eine Kurve von Grad $n$ hat Seitenlänge $2^n$ und somit $4^n$ Punkte. Der durchschnittliche Aufwand pro Punkt $avg\_Cost(n)$ lässt sich wie folgt berechnen: \ \\
\[avg\_Cost(n) = \frac{Op(n)}{4^n} = c \cdot \left(\frac{4^n}{3 \cdot 4^n} - \frac{1}{3 \cdot 4^n}\right) = \frac{1}{3}c - c \cdot \frac{1}{3 \cdot 4^n}\] 
\[\Rightarrow \lim_{n\to\infty} avg\_Cost(n) = \lim_{n\to\infty} \frac{Op(n)}{4^n} = \frac{1}{3}c\]  \ \\ 
Der durchschnittliche Rechenaufwand pro Punkt konvergiert also zu einem konstanten Wert. Somit erhöht sich der absolute Aufwand bei Erhöhung des Grads um eins ungefähr um den Faktor vier, der durchschnittliche Aufwand pro Punkt aber gleich bleibt, da sich die Anzahl der Punkte mit Erhöhung des Grads ebenso vervierfacht.

% TODO: Je nach Aufgabenstellung einen der Begriffe wählen
\section{Korrektheit}
Im Gegensatz zu Aufgabenstellungen, die Floating-Point Berechnungen enthalten oder im Bereich der Kryptographie liegen, ist in unserem Fall eine Überprüfung der Korrektheit unserer Lösung zumindest in Graden $<10$ allein schon durch das Betrachten der visuellen Repräsentation der Koordinaten im vom Rahmenprogramm erzeugten .svg File möglich. Automatisierte Tests können lediglich die erzeugten Koordinaten mit von einer anderen Implementierung berechneten Koordinaten abgleichen und feststellen, ob diese identisch sind, was nur bei einer bewiesen korrekten Vergleichsimplementierung aussagekräftig ist oder große Mengen an im Vorhinein gespeicherten Punktdaten erfordert.
Der Nachweis der Korrektheit der Algorithmen ist also zunächst nur durch detailliertes Nachvollziehen ihrer Funktionsweise möglich, kann aber in niedrigeren Graden sehr gut visuell überprüft werden. Insbesondere da Aufrufe unserer Algorithmen für höhere Grade die Ergebnisse niedrigerer Grade, dessen Korrektheit noch gut überprüfbar ist, dynamisch wiederverwenden, kann von einer insgesamten Korrektheit der Algorithmen ausgegegangen werden.

\FloatBarrier
\section{Performanzanalyse}
Um das Laufzeitzeitverhalten unserer Implementierung der Moore-Kurve zu überprüfen und beurteilen, haben wir Benchmarks folgender vier Implementierungen auf einem MacBook Pro mit einem 3,1 Ghz Dual-Core Intel i7 und 16 Gb 1867 Mhz DDR3 RAM durchgeführt.

\begin{center}
    \begin{tabular}{| r | r | r | r | r | r | r |}
    \hline
    Grad \textit{n} & \# Punkte & Speicher & \textit{asm\_avx} & \textit{asm} & \textit{c\_batch} & \textit{c\_iter}  \\ \hline
    3 & 64 & 512 B  & 27 ns & 23 ns & 67 ns & 564 ns \\  \hline
    6 & 4096 & 32 KiB  & 601 ns & 1213 ns & 6524 ns & 106 $\mu s$ \\  \hline
    9 & 26144 & 2048 KiB  & 179 $\mu s$ & 289 $\mu s$ & 830 $\mu s$ & 9436 $\mu s$  \\  \hline
    12 & 16,78 $\cdot 10^6$ & 128 MiB & 21,5 ms & 24,5 ms & 94,3 ms & 705,6 ms \\ \hline
    15 & 1,07 $\cdot 10^9$ & 8 GiB & 1575,9 ms & 1607,4 ms & 7108,1 ms & 50759 ms \\ \hline
    16 & 4,29 $\cdot 10^9$ & 32 GiB & 108,3 s & 125,7 s & 175,9 s & 442,8 s \\ \hline
    \end{tabular}
\end{center}

\begin{figure}[htbp] 
    \centering
    \subfloat[alle Implementierungen]{{\includegraphics[width=0.45\textwidth]{laufzeit_alle_1-9.png} }}%
    \qquad
    \subfloat[ohne \textit{c\_iter}]{{\includegraphics[width=0.45\textwidth]{laufzeit_ohne_iter.png} }}%
    \caption{Laufzeitverhalten der vier Implementierungen bis Grad 9}%
    \label{fig:Laufzeitvergleich}%
 \end{figure}
 
 \FloatBarrier
 \subsection{Iterative Berechnung der Moore-Koordinaten}
Aus Abb. \ref{fig:Laufzeitvergleich} geht klar hervor, dass die Punkt-für-Punkt-Berechnung der Moore-Koordinaten (wie in Sektion 2 beschrieben) deutlich ineffizienter ist als die anderen Implementierungen ist. Ein Vorteil dieser Variante ist zwar, dass sie nur sehr wenige Speicherzugriffe benötigt, da alle Berechnungen in den Registern durchgeführt werden können und nur das Endergebnis in den Speicher geschrieben wird. Durch das rein iterative Vorgehen kann dieser Algorithmus jedoch nicht sinnvoll auf Datenebene parallelisiert werden. Außerdem kann der dynamische Ansatz, von dem die übrigen Implementierungen profitieren, hier nicht angewandt werden. Im folgenden liegt daher der Fokus auf den restlichen Implementierungen.

\subsection{Vergleich Assembler- mit Referenzprogramm}
An Abb. \ref{fig:Laufzeitvergleich}.b und den Benchmarkwerten kann man erkennen, dass die Assembler Implementierungen deutlich performanter sind als die C-Referenzimplementierung. 
Für Grad 6 ist die Assembler-Implementierung, welche 128-bit Register benutzt, fünf mal schneller als das C-Programm. Dies ist darin begründet, dass sich die dynamische Berechnung der Hilbert-Kurve optimal durch SIMD parallelisieren lässt.
Somit können einige Koordinaten gleichzeitig innerhalb einer Operation verarbeitet werden. Die Anzahl der Punkte, die in ein SIMD-Register passen ist abhängig vom Datentypen der Koodinaten und der Größe der Vektorregister.
Wir haben uns entschieden, von der in der Aufgabenstellung vorgegebenen Signatur \texttt{moore(uint64\_t degree, uint64\_t *x, uint64\_t *y)} abzuweichen und 32 bit Integer zu benutzen. Der maximal darstellbare Wert verringert sich dadurch zwar auf $2^{32}-1$, jedoch werden größere Werte erst ab einem $n > 32$ benötigt. Der Speicherbedarf einer Moore-Kurve mit Grad 33 betrüge $4^{33} \cdot 2 \cdot 4 \text{ Bytes } \approx 590 \cdot 10^{18} \text{ Bytes} = 590 \cdot 10^6$ Terrabytes, wenn man alle Koordinaten als 32 bit unsigned Integers darstellen könnte und das Doppelte bei korrekter Benutzung von 64-bit Integern. Aus diesem Grund ist es praktisch unmöglich, alle Punkte der Moore-Kurve 33. Grades zu speichern und die Nutzung von 32-bit Integern anstelle von \texttt{uint64\_t} bringt realistisch gesehen keine Einschränkungen mit sich. \newline
Durch die Reduzierung auf 32-bit Intger können nun bei Verwenung von 128 bit Vektorregistern vier und bei 258-bit Registern acht Werte gleichzeitig verarbeitet werden (siehe Abb. \ref{fig:SIMD}).

\begin{figure}[h] 
    \centering
    \subfloat{{\includegraphics[width=0.4\textwidth]{hilbert_transformation.png} }}%
    \caption{Parallelisierung durch SIMD}%
    \label{fig:SIMD}%
\end{figure}

Durch Disassemblieren der mit \texttt{-march=native -O3} kompilierten Binärdatei haben wir festgestellt, dass der Compiler den C-Code nicht vektorisiert. Daher war unsere anfängliche Vermutung, dass \textit{asm\_128} (unsere Assembler-Implementierung, die 128-Bit Register Benutzt) etwa vier mal so schnell ist wie unsere C-Implementierung und \textit{asm\_256} (256-Bit AVX-Register) in der Hälfte der Zeit von \textit{asm\_128} terminiert.

\begin{figure}[h] 
    \centering
    \subfloat[Speedup Assembler- ggü. C-Implementierung]{{\includegraphics[width=0.45\textwidth]{speedup_base_c.png} }}%
    \qquad
    \subfloat[Speedup \textit{asm\_256} ggü. \textit{asm\_128}]{{\includegraphics[width=0.45\textwidth]{speedup_base_asm_128.png} }}%
    \caption{Laufzeitvergleich C/ASM}%
    \label{fig:Laufzeitvergleich Speedup}%
 \end{figure}
 
 Abb. \ref{fig:Laufzeitvergleich Speedup}.a bestätigt den ersten Teil unserer Vermutung. Zwischen Grad 4 und 15 ist die \textit{asm\_128} Implementierung 2,8 bis 5,3 mal schneller als die Referenzimplementierung in C. Der zweite Teil trifft nur teilweise zu. Wie in Abb. \ref{fig:Laufzeitvergleich Speedup}.b zu sehen ist, profitieren vor allem Grade zwischen 6 und 10 von den größeren Vektorregistern. Bei größeren Graden bietet die Verwendung von AVX-Registern keine signifikante Performanzsteigerung.
 \FloatBarrier
\subsection{Cache Effizienz}
Abb. \ref{fig:Laufzeitvergleich Speedup}.a wirft die Frage auf, weshalb die Assembler Implementierug bei Grad 6 besonders performant ist. Dies lässt sich vermutlich damit beantworten, dass die Moore-Kurve sechsten Grades genau \textit{32 KiB} an Speicher benötigt und die Größe des L1 Caches der Maschine, auf der getestet wurde \textit{32 KiB} beträgt. Um diese Annahme zu überprüfen wurde das Programm mit dem MacOS Tool \textit{Instruments} auf Cache-Misses (insbesondere L1) überprüft, mit dem Ergebnis, dass 6 der höchste Grad ist, bei dem noch nahezu keine L1 Misses auftreten.

\begin{figure}[h] 
    \centering
    \subfloat[Grad 1 - 15]{{\includegraphics[width=0.45\textwidth]{ns_per_coord_1-15.png} }}%
    \qquad
    \subfloat[Grad 1 - 16]{{\includegraphics[width=0.45\textwidth]{ns_per_coord_1-16.png} }}%
    \caption{Nanosekunden pro Punkt der Moore-Kurve}%
    \label{fig:Laufzeitvergleich Skaliert}%
 \end{figure}
 
 Der Vollständigkeit halber muss noch der Effizienzeinbruch bei Grad 16 angesprochen werden. Dieser ist damit zu erklären, dass die Speicheranforderung für die Berechnung der Moore-Kurve auf 32GiB steigt. Da unser Computer nur 16 GiB an Arbeitsspeicher zur Verfügung stellt, müssen nun häufig Speicherseiten auf die Festplatte ausgelagert werden, was zu starken Performanzeinbußen führt.


\FloatBarrier
\section{Zusammenfassung und Ausblick}
%Wir haben zwei Algorithmen vorgestellt: Einen, der die Berechnung eines einzelnen Punktes für gegebenen Index und Grad durchführt und einen, der eine effiziente Berechnung aller Koordinaten der gesamten Kurve eines gegebenen Grads ermöglicht.
Selten gibt es einen Punkt, ab dem man mit Sicherheit die Möglichkeit weiterer Performanzoptimierungen eines Programms ausschließen kann. Besonders beim Schreiben von Assemblycode können so viele Faktoren wie eine simple Änderung der Instruktionsreihenfolge die Schnelligkeit des Programms beeinflussen, sodass die Annahme eines voll optimierten Programms nie getroffen werden kann. \\
Nach wachsender Einarbeitung in die Thematik sowie Entwicklung und ständiger Verbesserung unserer Algorithmen fallen uns zu diesem Zeitpunkt noch weitere mögliche Wege der Optimierung ein.\\
Unsere Benutzung von 32-Bit Integern statt 64-Bit-Integern zur Darstellung der Koordinaten hat zu einer höheren Parallelisierbarkeit des Programms geführt. Mit hoher Wahrscheinlichkeit kann für Grade $\leq 16$ die dynamische Anpassung der Datenbreite auf 16 Bit zu einem weiteren Performanzgewinn führen. In Verbindung mit der Benutzung noch breiterer Register, zum Beispiel 512 Bit AVX Registern können Berechnungen noch stärker parallelisiert werden, wobei sich die  Limitierung derartiger Optimierungen schon bei der Erhöhung von 128 Bit auf 256 Bit gezeigt hat, mit teilweise kaum besserer Perfomanz trotz breiterer Register. \\
Auch das Experimentieren mit einer veränderten Datenstruktur, die x- und y-Koordinate eines Punktes nah beieinander speichert, könnte in bestimmten Fällen die Performance durch eine verbesserte Reihenfolge von Lade- und Schreibbefehlen auf den Speicher erhöhen. \\
All dies kann in unserer Herangehensweise jedoch nur den konstanten Aufwand pro Punkt verringern. Einen Algorithmus zu entwickeln, der sich für den gesamten Aufwand nicht in einer exponentiellen Laufzeitklasse befindet ist allein schon durch die exponentiell ansteigende Punktanzahl und somit einer exponentiell steigenden Anzahl an Schreibbefehlen unmöglich.
\FloatBarrier
\section{Anhang: Algorithmen}

\begin{algorithm}[h]
\scriptsize
\begin{algorithmic}
\caption{Berechne Koordinaten eines Punktes auf Moore-Kurve}
\label{moore:iterative}
\Function{berechneMooreKoordinate}{$index$, $degree$}
	\State finde Quadrant $q$ des Punkts q heraus ($q \in \{A, B, C, D\}$)
	\State $coord \leftarrow$ berechneHilbertKoordinate($index$, $degree-1$)
	\If {$q = A$} rotiere $coord$ um 90$^{\circ} \circlearrowleft$
	\ElsIf{$q = B$} rotiere $coord$ um 90$^{\circ}\circlearrowleft$ und verschiebe nach oben
		
	\ElsIf{$q = C$} rotiere $coord$um 90$^{\circ}\circlearrowright$ und verschiebe nach oben und nach rechts
	\ElsIf{$q = D$} rotiere $coord$ um  90$^{\circ} \circlearrowright$ und verschiebe nach rechts
	\EndIf
\EndFunction
\Function{berechneHilbertKoordinate}{$index$, $degree$}
	\State [...]
\EndFunction
\end{algorithmic}
\end{algorithm}

\begin{algorithm}[h]
\scriptsize
\begin{algorithmic}
\caption{Berechne Koordinaten eines gegebenen Punkts auf der Hilbert-Kurve}
\label{hilbert:iterative}
\Function{berechneHilbertKoordinate}{$index$, $degree$}
	\State $sidelength \leftarrow 2^{degree}$
	\State $i \leftarrow 1$
	\State $coord \leftarrow (0, 0)$
	\While {$i < sidelength$}
		\State ermittle Quadrant $q \in \{A, B, C, D\}$ mit den letzten zwei Bits von $index$:
		\State $A \iff '00'$, $B \iff '01'$, $C \iff '10'$, $D \iff '11'$
		\If{$q \in \{A, D\}$}
			\If{$q = D$}\Comment{falls sich Punkt in einem der beiden oberen Quadranten befindet}
			\State spiegle $coord$ an $y = i-1-x$ 
			\Else 
			\State spiegle $coord$ an $y=x$ Achse/Winkelhalbierender
			\EndIf
		\Else 
		\State verschiebe $coord$ nach oben, denn $q \in \{B, C\}$
		\EndIf
		\If{$q \in \{C, D\}$}
		\State verschiebe $coord$ nach rechts
		\EndIf
		\State $index \leftarrow \frac{index}{4}$ \Comment{zweimaliger rechts-Bitshift um nächste 2 Bits anzusehen}
		\State $i \leftarrow 2i$\Comment{Die Seitenlänge verdoppelt sich mit jeder Iteration}
	\EndWhile
	\State \Return $coord$
\EndFunction
\end{algorithmic}
\end{algorithm}


\begin{algorithm}[h]
\scriptsize
\begin{algorithmic}
\caption{Berechne alle Punkte der Moore-Kurve}
\label{moore:batch}
\Function{berechneMoorePunkte}{$degree$, int[] $x$, int[] $y$}
\State Koordinaten für $degree =1$ in $x$- und $y$-Arrays hardcoden (jeweils 4 Werte)
\If {$degree = 1$}
	\Return
\EndIf	
\State berechneHilbertPunkte($degree - 1$, int[] $x$, int[] $y$)\Comment{schreibt Pkt. der H. Kurve Grad $degree-1$ in die Arrays}
\State $len \leftarrow 2^{degree - 1} -1$ \Comment{Offset für Translation}
\State $quarter \leftarrow 4^{degree-1}$\Comment{Seitenlänge der Hilbert-Kurve quadriert $= {(2^{degree - 1}})^2$}
\For {$i \leftarrow 0$ to $quarter -1$}
\State Quadrant B: rotiere $i$-ten Punkt in A um 90$^{\circ}\circlearrowleft$ und verschiebe um $len$ nach oben
\State Quadrant C: rotiere $i$-ten Punkt in A um 90$^{\circ}\circlearrowright$ und verschiebe um $len$ nach oben und rechts
\State Quadrant D: kopiere $i$-ten Punkt in C und verschiebe um $len$ nach unten
\State Quadrant A: kopiere $i$-ten Punkt in B and verschiebe um $len$ nach unten
\EndFor
\EndFunction
\ \\
\Function{berechneHilbertPunkte}{$degree$, int[] $x$, int[] $y$}
\State $len \leftarrow 2$ \Comment{Offset beginnt bei 2}
\State $quarter \leftarrow 4$\Comment{Algorithmus beginnt mit 4 hardgecodeten Punkten}
\For{$j \leftarrow 2 $ to $ degree$}\Comment{Die Erzeugung beginnt bei Grad 2, da Grad 1 bereits im Speicher steht}
\For{$i \leftarrow 0$ to $quarter-1$}                    
	\State Quadrant B: kopiere und verschiebe $i$-ten Punkt in A um $len$ nach oben
	\State Quadrant A: spiegle $i$-ten Punkt in A an Winkelhalbierender $x=y$
	\State Quadrant C: kopiere $i$-ten Punkt in B und verschiebe um $len$ nach oben
	\State Quadrant D: rotiere $i$-ten Punkt in A um 180$^{\circ}$ und verschiebe um $len$ nach rechts
\EndFor
\State $len \leftarrow 2*len$\Comment{mit jeder Iteration verdoppelt sich die Seitenlänge..}
\State $quarter \leftarrow 4*quarter$\Comment{...und vervierfacht sich die Anzahl der Punkte}
\EndFor
\EndFunction
\end{algorithmic}
\end{algorithm}



% TODO: Fuegen Sie Ihre Quellen der Datei Ausarbeitung.bib hinzu
% Referenzieren Sie diese dann mit \cite{}.
% Beispiel: CR2 ist ein Register der x86-Architektur~\cite{intel2017man}.
\nocite{intel2017man}
\nocite{moore1900}
\nocite{PointForPoint}
\bibliographystyle{plain}
\bibliography{Ausarbeitung}{}

\end{document}
