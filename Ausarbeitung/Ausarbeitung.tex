% Diese Zeile bitte -nicht- aendern.
\documentclass[course=erap]{aspdoc}

%%%%%%%%%%%%%%%%%%%%%%%%%%%%%%%%%
%% TODO: Ersetzen Sie in den folgenden Zeilen die entsprechenden -Texte-
%% mit den richtigen Werten.
\newcommand{\theGroup}{142} % Beispiel: 42
\newcommand{\theNumber}{A213} % Beispiel: A123
\author{Anton Baumann \and Felix Brandis \and Michal Cizevskij}
\date{Sommersemester 2020}  % Beispiel: Wintersemester 2019/20
%%%%%%%%%%%%%%%%%%%%%%%%%%%%%%%%%

% Diese Zeile bitte -nicht- aendern.
\title{Gruppe \theGroup{} -- Abgabe zu Aufgabe \theNumber}

\begin{document}
\maketitle

\section{Einleitung}


\section{Lösungsansatz}

\subsection{Die Moore Kurve}

Die Moore Kurve ist eine flächenfüllende Kurve, 
....\ \\
Das Problem kann also umformuliert werden in die Erzeugung der Hilbert Kurve und dem anschließendenden Kopieren, Verschieben und Rotieren der Punkte. 
\\
Es gibt rekursive Algorithmen für die Erzeugung der Hilbert Kurve, diese klammern wir im Folgenden jedoch aus. (Siehe Aufgabenstellung)

\subsection{Erzeugung der Hilbert Kurve: iterativer Punkte-für-Punkt Ansatz}
%rekursiven Ansatz vermieden





\subsection{Erzeugung der Hilbert Kurve: Bottom-up Ansatz}

\subsection{Laufzeitanalyse und Vergleich}




%Für die Erzeugung der Koordinaten der Moore Kurve eines gegebenen Grades gibt es verschiedene Herangehensweisen. Allen im Folgenden dargestellten Ansätzen liegt die Tatsache zugrunde, dass die Moore Kurve des Grads n ($ \forall  n \geq 2$) aus 4 rotierten, gespiegelten, verschobenen und dann zusammengefügten Kopien der weitaus besser bekannten Hilbert Kurve zusammengesetzt werden kann. (siehe Bild 1) 

%Ein Algorithmus, der für gegebenen Grad $n$ und Index $i$ ein Koordinatentupel $(x, y)$, das die Lage des $i$-ten Punktes beschreibt, als Ausgabe liefert, kann also naiv folgendermaßen implementiert werden:

% TODO: Je nach Aufgabenstellung einen der Begriffe wählen
\section{Korrektheit/Genauigkeit}


\section{Performanzanalyse}



\section{Zusammenfassung und Ausblick}

% TODO: Fuegen Sie Ihre Quellen der Datei Ausarbeitung.bib hinzu
% Referenzieren Sie diese dann mit \cite{}.
% Beispiel: CR2 ist ein Register der x86-Architektur~\cite{intel2017man}.
\bibliographystyle{plain}
\bibliography{Ausarbeitung}{}

\end{document}
